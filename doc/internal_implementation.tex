\chapter{Implementation details}
\section{The Program Monitor}
\subsection{High-level Overview}
The Program Monitor can be termed the core of the data collection system in Aesalon. It is responsible for
launching the process, loading the various collection modules, parsing the executable's ELF and DWARF data,
and loading the polisher modules.

\subsection{Command-line invocation}
Provided, in alphabetical order, are the command-line options for the program monitor. The general form
for invocation is \texttt{aesalon [options] [--] filename [arguments]}.
\begin{description}
 \item[\texttt{--help:}] Displays an automatically-generated list of options, with brief descriptions.
 \item[\texttt{--logFile:}] If set, specifies a file to write events to. \texttt{.alog} is the standardized extension.
 \item[\texttt{--modules:}] A colon-separated list of modules to load.
 \item[\texttt{--networkWait:}] The number of network clients to wait for a connection from before commencing
execution.
 \item[\texttt{--shmSize:}] The size of the SHM to use; must be a multiple of the page size of the operating system.%
  \footnote{Typically 4096 bytes on x86 and x86\_64 Linux systems.}
 \item[\texttt{--tcpPort:}] The TCP port to listen on for client connections.
\end{description}

\subsection{Environment variables}
Aesalon's behaviour is also influenced by a number of environment variables.
\begin{description}
 \item[\texttt{AesalonSearchPath:}] A colon-separated list of paths to search for module directories.
\end{description}

\subsection{Configuration}
There are five different levels of configuration, presented in order of processing.
\begin{description}
 \item[Module configuration] applies a default range of settings on a module-wide basis. These files are located in the
module root directories, called \texttt{monitor.conf}.
 \item[Global configuration] is the system-wide defaults. The default path is \texttt{/etc/default/aesalon.conf.}
 \item[User configuration] is the user-wide defaults. The default path is \texttt{$\sim$/.config/aesalon/aesalon.conf.}
 \item[Local configuration] is the directory. This is a special case; any file named (by default) \texttt{.aesalon.conf}
  will be parsed and applied to any instances launched from that directory.
 \item[Command-line arguments] are the final level of configuration. These override all other configurations, and are
specified
  on the command-line.
\end{description}

\begin{figure}[t]
  \small{\texttt{%
set shmSize=1048576 \\
set tcpPort=6321 \\
\\
module dynamicMemory \\
set rttiTracking=false \\
set polisher=/tmp/build/dynamicMemory\_polisher-test.so
  }}
  \caption{Example global configuration file.\label{exampleconfigfile}}
\end{figure}

In configuration files, options are specified as demonstrated in Figure \ref{exampleconfigfile}; options are
specified
in the same manner on the command-line. The general form of a configuration item is \texttt{set <item>=<content>};
items may be specified inside a namespace, or a module. If no module command is specified, then the default module,
called ``global'', is used.

Note that there is currently no method of specifing configuration options; thus, no warnings about mis-spelled arguments
or configuration items are given.

\section{Design}
\subsection{Source layout}
The monitor is designed in a somewhat logical manner. The source is separated into directories, each of which handles a
specific set of duties.

\begin{description}
 \item[.:] contains the ``essential'' files; the generic, global sources, if you will.
 \item[program:] contains source files relating to execution control and executable analysing.
 \item[network:] has all the network-related source files.
 \item[module:] the monitor module-related source code.
 \item[polisher:] the polisher module interface; implementations only.
 \item[misc:] any miscellaneous/utility classes.
\end{description}

\subsection{Threads}

There are two primary threads in the monitor:
\begin{description}
 \item[\textnormal{The} main thread] handles all initialization (including executable parsing), then eventually calls
\emph{wait()} on the monitored process.
 \item[\textnormal{The} reader thread] reads from the SHM, processes the data packets via the monitor module, logs the
packets to a log file if applicable, then sends the packets via the network socket to any connected visualizers.
\end{description}
